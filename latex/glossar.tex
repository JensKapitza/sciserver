
\newdualentry{dms} % label
{DMS}            % abbreviation
{Datenmanagementsystem}  % long form
{System zum Verwalten von Dateien, in dem auch gesucht werden kann.} % description
 
\newdualentry{p2p} % label
{P2P}            % abbreviation
{Peer-to-Peer}  % long form
{Kommunikationsmodell, in dem jeder Teilnehmer mit jedem anderen reden kann.} % description
 
\newdualentry{nfs} % label
{NFS}            % abbreviation
{Network File System}  % long form
{Bei Linux gleichnamiger Dienst, generell ist ein Netzwerkdienst gemeint, welcher sein Dateisystem über einen Computer hinweg zur Verfügung stellt.} % description
 
\newdualentry{cifs} % label
{CIFS}            % abbreviation
{Common Internet File System}  % long form
{Netzwerk Kommunikationsprotokoll, unter Windows und durch Samba unter Linux bereitgestellter Dienst zum Datenaustausch zwischen Computern.} % description

\newdualentry{ads} % label
{ADS}            % abbreviation
{alternative data streams}  % long form
{Windows Funktion zur Datenspeicherung innerhalb einer Datei} % description
 
\newdualentry{rf} % label
{RF}            % abbreviation
{resource forks}  % long form
{Eine Technologie, die von Apple verwendet wurde, zur Datenspeicherung innerhalb einer Datei.} % description

\newdualentry{lvm} % label
{LVM}            % abbreviation
{Logical-Volume-Manager}  % long form
{Ein LVM, bietet einen virtuellen Zusammenschluss mehrerer Festplatten zu einem logischen Laufwerk. } % description



\newdualentry{rpc} % label
{RPC}            % abbreviation
{Remote Procedure Call}  % long form
{Aufruf einer fernen Prozedur} % description



\newdualentry{nat} % label
{NAT}            % abbreviation
{Network Address Translation}  % long form
{Verfahren zum Ersetzen der Adressinformationen in IP Datenpaketen} % description


\newdualentry{jms} % label
{JMS}            % abbreviation
{Java Message Service}  % long form
{Bibliothek zum Kommunizieren durch Nachrichtenaustausch. } % description
 
 \newdualentry{pae} % label
 {PAE}            % abbreviation
 {Physical Address Extension}  % long form
 {Erweiterung der Adressierung in 32 Bit Architektur} % description
 
 
\newdualentry{cdi} % label
{CDI}            % abbreviation
{Contexts and Dependency Injection}  % long form
{Aus Java EE, Möglichkeit zur Auflösung von Objektabhängigkeiten zur Laufzeit sowie Verwaltung von Lebenszyklen verwalteter Objekte innerhalb einer Anwendung.} % description

\newdualentry{jpa} % label
{JPA}            % abbreviation
{Java Persistence API}  % long form
{API in Java zur Verwendung einer Datenbank, diese API beinhaltet auch eine \acrshort{orm} Abbildung von Klassen auf Tabellen der Datenbank.} % description
 
\newdualentry{orm} % label
{ORM}            % abbreviation
{Object-Relational Mapping}  % long form
{Bidirektionale Abbildung der Spalteninformationen einer Datenbank in Eigenschaften eines Objektes.} % description

\newdualentry{tls} % label
{TLS}            % abbreviation
{Transport Layer Security}  % long form
{Verschlüsselung durch Zertifikate, häufige Verwendung bei HTTPS, SMTPS, FTPS. Sollte nicht mit SFTP verwechselt werden, welches ein anderes Verfahren nutzt (Schlagwort: SSH).} % description
 

\newdualentry{ide} % label
{IDE}            % abbreviation
{Integrierte Entwicklungsumgebung}  % long form
{Eine IDE erlaubt das einfachere Schreiben von Quellcode und unterstützt verschiedene Tools, die das Arbeiten z. B. mit Datenbanken vereinfachen.} % description




\newdualentry{ttl} % label
{TTL}            % abbreviation
{Time to live}  % long form
{Lebenszeit, z. B. für ein Verwerfen eines Paketes im Netzwerk.} % description


 
\newdualentry{fqdn} % label
{FQDN}            % abbreviation
{Fully Qualified Domain Name}  % long form
{Absoluter vollständiger Domainname eines Computers.} % description
  
 
 
 
 \newdualentry{stun} % label
 {STUN}            % abbreviation
 {Session Traversal Utilities for NAT}  % long form
 {Möglichkeit zum Erkennen der externen Verbindungsparameter sowie den \acrshort{nat} Firewall-Typen für anfragende Geräte.} % description
 
  
 
 \newdualentry{acc} % label
 {ACC}            % abbreviation
 {Application Client Container}  % long form
 {Spezielle Form eines Clients, der komplett vom Anwendungsserver abhängig ist.} % description
 
 \newdualentry{KEY} % label
 {ABBR}            % abbreviation
 {LONG}  % long form
 {DESC} % description
 
 
 
 